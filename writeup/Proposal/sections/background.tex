In this section, we review prior work on resource disaggregation,
resource reconfiguration, and OS reconfiguration, and contrast
this with our proposed research.

\subsection{Resource disaggregation}
The concept of resource disaggregation is that instead of building
servers by collecting all required hardware components within a
physical box, different hardware components can be placed into
multiple resource blades, communicating with each other via network
switches (Figure~\ref{fig:archi}). Compared to the traditional
monolithic architecture, a disaggregated architecture has the
potential to improve
resource utilization, supports hardware upgrade and scaling,
and supports heterogeneous computing.  Commercial prototypes include
Intel’s Rack Scale Architecture~\cite{Intel_RSA}, HP’s “The
Machine”~\cite{HP_The_Machine}, Facebook’s Disaggregated
Rack~\cite{FB_disaggregated_rack}, and IBM’s Composable
System~\cite{chung2018towards}.  Academic proposals include
dRedBox~\cite{katrinis2016rack}, Firebox\cite{asanovic2014firebox},
and disaggregated memory blade~\cite{lim2009disaggregated}.

Several systems have been built that leverage disaggregated memory
to optimize memory-intensive applications, either by remote memory
swapping~\cite{lagar2019software, al2020effectively, gu2017efficient}
or by exposing a remote memory abstraction to
applications~\cite{aguilera2018remote}.  LegoOS~\cite{shan2018legoos}
proposes the \emph{splitkernel} architecture and implements an OS
especially built for disaggregated hardware. Other work
evaluates the networking aspect of achieving resource
disaggregation\cite{abali2015disaggregated, gao2016network}.

While these systems explore disaggregated resources, the
operating systems themselves assume that, once booted, the
collection of hardware resources available does not change.
We propose to develop an OS where the set of hardware resources
that it uses is highly fluid and governed by application demand
rather than by hardware limitations.

\subsection{Resource reconfiguration}
In cloud computing, resource \emph{vertical scaling} (aka
\emph{scale-up}) and resource overcommitment raise the need for
resource reconfiguration.  Most prior research focuses on memory
overcommitment and auto-scaling and use memory
ballooning~\cite{waldspurger02} to adjust the memory accessible to
VMs~\cite{salomie2013application, amit2014vswapper, hines2011applications,
agmon2014ginseng, shaikh2015dynamic, molto2013elastic}. These
approaches still have a limited amount of computing resources that
are specified when a VM is booted. We propose to eliminate these
limitations and achieve a high degree of resource reconfiguration.

Similar to our approach are the Resource-as-a-Service cloud
model~\cite{ben2012resource} and nonkernel~\cite{ben2013nonkernel},
which also propose to build an OS that views computing resources
in a fine-grained fashion.  However, these systems target traditional
monolithic servers, while we believe that a disaggregated architecture
would be a better target.


\subsection{Reconfiguration policies}
\cite{sedaghat2013virtual} describes various VM repacking
policies that take either changes in workload or the
total load as input and output a decision on whether to reconfigure,
and if so, the new optimal resource distribution.
For memory-intensive workloads, a study of cloud providers shows that
auto-scaling of the memory resources can be accomplished by analyzing the
applications' memory miss ratio curve (MRC) and modeling it as a
hyperbola~\cite{novak2020auto}.

Prior research on web applications~\cite{yazdanov2014lightweight}
shows that fine-grained vertical scaling of multi-tiered web
applications can be accomplished through online reinforcement learning
models.  Additionally, with the recent development of machine
learning models, prediction of the optimal amount of
resources can also be attained by certain supervised deep learning
models.  Since the input is a series of time-correlated data,
recurrent neural networks can capture the important
features~\cite{gers1999learning}.

\cite{farokhi2016hybrid} suggests to use a control-theoretic
approach to decide how to vertical scale resources in web-facing
applications.
They propose a hybrid between two specific approaches:
the performance-based (PC) approach and capacity-based (CC) approach.
PC directly optimizes application performance while CC focuses on
optimizing resource utilization.  The paper shows that
combining these two approaches can combine the benefits.

\subsection{OS reconfiguration}
Prior work on OS reconfiguration has focused on updating the
policies or mechanisms of a deployed OS. The main purpose is to
carry out application-specific optimizations, insert third-party
modules, enable dynamic monitoring, or fix bugs on the
fly~\cite{soules2003system, baumann2007reboots, baumann2005providing,
chen2006live, chen2007polus}. Learned OS~\cite{zhang2019learned}
proposes using machine learning to tune and build an OS. We propose
that, in addition to policies and mechanisms, OS reconfiguration
should also include the ability to reconfigure the computing resources
an OS can use.  We are interested in both machine learning and more
traditional heuristics to achieve better resource utilization.
